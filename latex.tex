\documentclass[]{amsart}

\usepackage[draft]{hyperref} % remove all hyperlinks included in the translation
\usepackage{amsmath}
\usepackage{amssymb}
\usepackage%[demo]
{graphicx}
\usepackage[spanish]{babel}
\usepackage{letltxmacro}
\usepackage{float}
\floatplacement{figure}{H} % do not float images
% ----------------------------------------------------------------
\vfuzz2pt % Don't report over-full v-boxes if over-edge is small
\hfuzz2pt % Don't report over-full h-boxes if over-edge is small
%
\newtheorem{theorem}[equation]{Teorema}
\newtheorem{corollary}[equation]{Corolario}
\newtheorem{lemma}[equation]{Lemma}
\newtheorem{proposition}[equation]{Proposición}
\theoremstyle{definition}
\newtheorem{definition}[equation]{Definición}
\theoremstyle{remark}
\newtheorem{remark}[equation]{Observación}
\newtheorem{example}[equation]{Ejemplo}
\newtheorem{exercise}[equation]{Ejercicio}
\newtheorem{watch}[equation]{Advertencia}

\numberwithin{equation}{subsection}

\numberwithin{section}{part}

\providecommand{\tightlist}{%
	\setlength{\itemsep}{0pt}\setlength{\parskip}{0pt}}

% Fix pandoc not parsing inside latex blocks

\let\Begin\begin
\let\End\end

% Resizing all images

\LetLtxMacro{\OldIncludegraphics}{\includegraphics} 
\renewcommand{\includegraphics}[1]{\OldIncludegraphics[width=0.5\textwidth]{#1}}

%opening
\title{Estructuras algebraicas}
\author{Fernando Muro}
\address{Universidad de Sevilla,
	Facultad de Matem\'aticas,
	Departamento de \'Algebra,
	Avda. Reina Mercedes s/n,
	41012 Sevilla, Spain}
\email{fmuro@us.es}
\urladdr{http://personal.us.es/fmuro}


\begin{document}

\maketitle

\begin{abstract}
Notas de la asignatura Estructuras Algebraicas del Grado en Matemáticas de la Universidad de Sevilla. Esta versión en PDF se ha generado automáticamente a partir de la página web \url{http://asignatura.us.es/estalg/} y puede contener errores derivados del proceso de conversion. Les agradecería que informasen de cualquier error que encuentren escribiendo a \texttt{fmuro@us.es}.
\end{abstract}

\tableofcontents

\part{Anillos}

\input{hugotopdf/outputs/rings/_index.tex}

\section{Definiciones}

\input{hugotopdf/outputs/rings/definitions.tex}

\section{Factorización}

\input{hugotopdf/outputs/rings/factorization.tex}

\part{Módulos}

\input{hugotopdf/outputs/modules/_index.tex}

\part{Cuerpos}

\input{hugotopdf/outputs/fields/_index.tex}

\section{Extensiones}

\input{hugotopdf/outputs/fields/extensions.tex}

\section{Teoría de Galois}

\input{hugotopdf/outputs/fields/Galois.tex}

\end{document}
